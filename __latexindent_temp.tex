
@article{doi:10.3102/0013189X013006004,
author = {Benjamin S. Bloom},
title ={The 2 Sigma Problem: The Search for Methods of Group Instruction as Effective as One-to-One Tutoring},
journal = {Educational Researcher},
volume = {13},
number = {6},
pages = {4-16},
year = {1984},
doi = {10.3102/0013189X013006004},

URL = { 
        https://doi.org/10.3102/0013189X013006004
    
},
eprint = { 
        https://doi.org/10.3102/0013189X013006004
    
}

}

@article{doi:10.3102/0034654317751919,
author = {Jean Stockard and Timothy W. Wood and Cristy Coughlin and Caitlin Rasplica Khoury},
title ={The Effectiveness of Direct Instruction Curricula: A Meta-Analysis of a Half Century of Research},
journal = {Review of Educational Research},
volume = {88},
number = {4},
pages = {479-507},
year = {2018},
doi = {10.3102/0034654317751919},

URL = { 
        https://doi.org/10.3102/0034654317751919
    
},
eprint = { 
        https://doi.org/10.3102/0034654317751919
    
}
,
    abstract = { Quantitative mixed models were used to examine literature published from 1966 through 2016 on the effectiveness of Direct Instruction. Analyses were based on 328 studies involving 413 study designs and almost 4,000 effects. Results are reported for the total set and subareas regarding reading, math, language, spelling, and multiple or other academic subjects; ability measures; affective outcomes; teacher and parent views; and single-subject designs. All of the estimated effects were positive and all were statistically significant except results from metaregressions involving affective outcomes. Characteristics of the publications, methodology, and sample were not systematically related to effect estimates. Effects showed little decline during maintenance, and effects for academic subjects were greater when students had more exposure to the programs. Estimated effects were educationally significant, moderate to large when using the traditional psychological benchmarks, and similar in magnitude to effect sizes that reflect performance gaps between more and less advantaged students. }
}

@book{gatto2017dumbing,
  title={Dumbing Us Down: The Hidden Curriculum of Compulsory Schooling},
  author={Gatto, John Taylor},
  year={2017},
  publisher={New Society Publishers}
}

@article{gupta2000liberating,
  title={Liberating Education from the Chains of Imperialism},
  author={Gupta, Pawan},
  journal={Learning Societies: A Reflective and Generative Framework. Udaipur: Shikshantar—The People's Institute for Rethinking Education and Development},
  year={2000}
}
